\documentclass[a4paper,11pt,titlepage,uplatex]{jsarticle}


% 数式
\usepackage{amsmath,amsfonts}
\usepackage{bm}
% 画像
\usepackage[dvipdfmx]{graphicx}
\usepackage{here}

\usepackage[hang,small,bf]{caption}
\usepackage[subrefformat=parens]{subcaption}
\captionsetup{compatibility=false}


\begin{document}

\title{炉物理パラメータ不確かさ評価のための複数の模擬パラメータを活用した無次元化CV-S法}
\author{鷹見大地}
\date{\today}
\maketitle

\tableofcontents
\clearpage

\part{序論}
\section{背景}
工学の分野において、数値計算によってシステムの特性値•パラメータを予測することが一般的に行われている。
計算モデルや計算手法•アルゴリズムの高度化によって数値計算の予測値の精度を向上させることは重要であるが、
それと同時に、予測値が内包する不確かさを定量化することも、計算結果の信頼性の向上や効率的な研究開発の実現などの観点から重要である。
原子炉工学の分野、特に原子炉物理の分野において、不確かさの定量化に関する研究が、ここ10年、活発に行われてきている。

原子炉物理分野において、数値計算の予測値の不確かさの要因はいくつかあるが、入力情報の1つとして用いられる
核データの不確かさが主なものの一つとして挙げられる。
それを受けて、各データの不確かさに起因する炉物理計算結果の不確かさの定量化が、核データ工学、原子炉物理を跨いだ重要な研究課題となっている。

核データの不確かさに起因する炉物理パラメータの不確かさを評価する方法は、感度係数を用いて誤差伝播を計算する方法と
ランダムサンプリング法を用いる方法とに大別される。前者は、感度係数を得るために摂動理論に基づく複雑な計算を行う必要があるが、
感度係数さえ得ることができれば計算時間を要さないという利点がある。ただし、感度係数は出力(炉物理パラメータ)と
入力(核データ)に対する一時微係数であることから、出力と入力に線形性が仮定されるため、厳密な誤差伝播計算を行うことができないという問題がある。
一方、後者は、前者の方法における線形性を仮定しない一方で、膨大な計算時間を要するという短所がある。
また、ランダムサンプリング法で得られる結果(不確かさ)には必ず統計的な不確かさが含まれることになり、これを小さくするためには表本数を大きくとらなければいけない点にも留意する必要がある。

これまでに、少ない標本数で不確かさの小さい結果を得るための効率的なランダムサンプリング手法として、ラテン超方各法や、決定論的サンプリング手法などが提案されてきている。
一方、我々の研究グループでは、感度係数を計算するためのコードを開発していることから、制御変量法$\left(Control variate法、CV法\right)$と呼ばれる方法と
感度係数を組み合わせて利用する方法$\left(CV-S法\right)$を考案し、核燃料の燃焼問題においてその有効性を示した。
$CV法$とは、評価対象とする出力パラメータ$\left(対象パラメータ、ターゲットパラメータ\right)$に対して、振る舞いが類似であり、
その統計量が既知もしくは高精度な推定が容易な別なパラメータ$\left(模擬パラメータ、モックアップパラメータ\right)$を考え、
その2つのパラメータの差分に着目することで、少ない標本数で平均値を効率的に計算する方法である。



\end{document}